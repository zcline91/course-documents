\usepackage{xstring}

% Change these commands to suit your needs.
\newenvironment{problems}{\begin{description}}{\end{description}}
\newenvironment{subproblems}{\begin{itemize}}{\end{itemize}}

% There should be 2 commands for each problem sourch that will be used
% in this document type, one of the form \{source}prob and the other of 
% the form \{source}subprob, where {source} is as the source appears in the 
% problems directory. These commands should take 1 argument, which is 
% the relative path to the problem in problems/{source}. These commands
% will be called and printed in the autogenerated files.

%%%%%%%%%%%%%%%%%%%%%%%%%%%%%%%%%%%%%%%%%%%%%%%%%%%%%%%%%%%%%%%%%%%%%%%%%
% EXAMPLES (rely on the use of the package xstring)
% Replace these with whatever commands you need for your problem sources.
\newcommand{\textbookprob}[1]{
  \pitem[\StrSubstitute{#1}{/}{.}]{textbook/#1}{solution}
}
  
\newcommand{\textbooksubprob}[1]{
  \StrCount{#1}{/}[\strcount] % count number of / in #1
  \StrBehind[\strcount]{#1}{/}[\probpart] % get what's right of final /
  \pitem[(\probpart)]{textbook/#1}{solution}
}
%%%%%%%%%%%%%%%%%%%%%%%%%%%%%%%%%%%%%%%%%%%%%%%%%%%%%%%%%%%%%%%%%%%%%%%%%
