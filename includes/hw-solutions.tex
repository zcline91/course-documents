\usepackage{xstring}

% Change these commands to suit your needs.
\newenvironment{problems}{\begin{description}}{\end{description}}
\newenvironment{subproblems}{\begin{itemize}}{\end{itemize}}

% If problems have parts, there should be 2 commands below for each 
% problem source that will be used in this document type, one of the form 
% \<source>prob and the other of the form \<source>subprob, where <source> 
% is as the source appears in the problems directory. These commands should 
% take 1 argument, which is the relative path to the problem in  the 
% directory problems/<sourc>. These commands will be called and printed 
% in the autogenerated files.

%%%%%%%%%%%%%%%%%%%%%%%%%%%%%%%%%%%%%%%%%%%%%%%%%%%%%%%%%%%%%%%%%%%%%%%%%
% EXAMPLES (rely on the use of the package xstring)
% Replace these with whatever commands you need for your problem sources.

% This will create an item whose 
% - label is created by taking the path to the problem in the 'textbook' 
%   subdirectory of the 'problems' directory, and replacing the / 
%   characters with . characters.
% - contents are the 'solution.tex' file in the path of the problem
\newcommand{\textbookprob}[1]{
  \pitem[\StrSubstitute{#1}{/}{.}]{textbook/#1}{solution}
}

% This will create an item whose
% - label is created by taking the last part of the path to the problem
%   and surrounding it in parentheses (e.g. if the path (argument 1) is 
%   '4/2/a', then the label will be '(a)')
% - contents are the 'solution.tex' file in the path of the problem
\newcommand{\textbooksubprob}[1]{
  \StrCount{#1}{/}[\strcount] % count number of / characters in #1
  \StrBehind[\strcount]{#1}{/}[\probpart] % get what's right of final /
  \pitem[(\probpart)]{textbook/#1}{solution}
}
%%%%%%%%%%%%%%%%%%%%%%%%%%%%%%%%%%%%%%%%%%%%%%%%%%%%%%%%%%%%%%%%%%%%%%%%%
