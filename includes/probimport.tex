\usepackage{import}
\usepackage{etoolbox}

% In the document that uses these commands, `\rootdir' must be defined before
% this file is called, and point to the root directory (the one containing 
% documents, problems, includes, scripts, etc.). This will usually be set to 
% something like ../../..


% generic command to import the contents of problems/#1/#2.tex, where #1 can 
% be a nested subdirectory
\newcommand{\pimport}[2]{\subimport{\rootdir/problems/#1}{#2}}

% -- modified \question command --
% There is one required argument (the path as in \pimport) and one optional
% argument (the point value of the question).
\newcommand{\pquestion}[2][]{%
  \ifblank{#1}{\question}{\question[#1]}%
  \pimport{#2}{statement}%
}

% -- modified \question command with a solutionbox --
% This is the same as \question but with a second required argument of the 
% height of the space reserved for the solution.
\newcommand{\pquestionsol}[3][]{%
  \pquestion[#1]{#2}%
  \begin{solutionbox}{#3}%
    \pimport{#2}{solution}%
  \end{solutionbox}%
}

% -- modified \part command --
% Similar to the \pquestion command with same arguments, but should be in a 
% \parts environment instead of a \questions environment.
\newcommand{\ppart}[2][]{%
  \ifblank{#1}{\part}{\part[#1]}%
  \pimport{#2}{statement}%
}

% -- modified \part command with a solutionbox --
% Similar to the \pquestionsol command with same arguments, but should be in a
% \parts environment instead of a \questions environment.
\newcommand{\ppartsol}[3][]{%
  \ppart[#1]{#2}%
  \begin{solutionbox}{#3}%
    \pimport{#2}{solution}%
  \end{solutionbox}%
}

% -- modified \subpart command --
% Similar to \pquestion and \ppart, but should be in \subparts environment.
\newcommand{\psubpart}[2][]{%
  \ifblank{#1}{\subpart}{\subpart[#1]}%
  \pimport{#2}{statement}%
}

% -- modified \subpart command with a solutionbox --
% Similar to \pquestionsol and \ppartsol, but should be in \subparts
% environment.
\newcommand{\psubpartsol}[3][]{%
  \psubpart[#1]{#2}%
  \begin{solutionbox}{#3}%
    \pimport{#2}{solution}%
  \end{solutionbox}%
}

% -- modified \item command --
% Imports the contents of problems/#2/#3.tex into an \item, where #2 and 
% #3 are the two required arguments. #1 is an optional argument for the 
% label of the \item.
\newcommand{\pitem}[3][]{%
  \ifblank{#1}{\item}{\item[#1]}%
  \pimport{#2}{#3}%
}